
\chapter{Conclusion\label{ch:conclusion}}

The current paper models the determinants of entrepreneurship entry decisions at the U.S. level, asking whether and how men and women respond differently to factors of interest. We expand on existing literature addressing gender asymmetries in self-employment, individual attributes and their effect on entry rates, as well as work on local environments and their role in fostering entrepreneurship. The probability of an agent becoming an entrepreneur is estimated via a probit model, as a function of socio-economic factors, demographic controls, local business climate, liquidity constraints and occupational choice. For every model, errors are clustered by metropolitan area, and coefficient equality is tested across gender groups.

When considering men and women's self-employment choices in the United States from 1998 to 2014, we find variation on a number of important dimensions in their motivations to start a business. At the individual level, availability of capital remains an important positive factor, along with skill versatility and educational attainment. Minority status and being married are significant deterrents, with similar effects for men and women. The differences occur when looking at prefered lifestyle choices, with female self-employment being associated with more hours worked and male self-employment, with a smaller time commitment. 

With regards to external factors, we find women are incentivized to a greater extent by tax credits and the state of the political climate, while men respond to positive changes in the local economy. We relate this finding to behavioral science literature highlighting different perceptions of risk and decision making under uncertainty for men and women\footnote{\cite{adams2012beyond}\cite{koellinger2013gender}}, and encourage future research in this respect. Across industries, we find a pattern of female underrepresentation in STEM fields, similar to the one in wage-employment, which disproves the hypothesis of a breaking-barriers mechanism to starting one's business. 

The need to understand and adapt to the changing demographics of the United States\footnote{One of the most recent projections of the U.S. Census Bureau for the future U.S. population denotes that "by 2044, more than half of all Americans are projected to belong to a minority group (any group other than non-Hispanic White alone); and by 2060, nearly one in five of the nation's total population is projected to be foreign born. + cite Colby, S. and Ortman, J.M. (2015)} and that of gender integration in the labor market are topics closely related to the goal of maintaining competitiveness on the global scale. To that end, the United States economy might not be able to afford productivity deterrents and unexploited growth sources due to persistent barriers for women and minorities. In conclusion, our work suggests that both empirically and theoretically, the factors affecting women's employment choices, particularly those of starting a business should become a priority that informs better-targeted policies. 