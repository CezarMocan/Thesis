\section{Future Work}

Our models were able to capture a significant proportion of the variability in self-employment entry, with Pseudo $R^2$ coefficients surpassing 0.5. There are however, areas that could benefit from further exploration, a more comprehensive set of variables being expected to add to the current research in insightful ways. 

In terms of education and its effect on self-employment rates, it is known that in most OECD countries men and women reach similar attainment levels\footnote{\cite{charles2002equal}}. However, there is a persistent gender gap in \textit{educational choices} across genders, with women being a majority in fields like health and education, and underrepresented in more technical programs, like hard sciences and engineering\footnote{\cite{charles2002equal}}. Given that our data doesn't capture field of study, but only level of educational attainment and industry of choice, one can suspect that similar education level might not bring the same effects across individuals, especially since technical fields are more often associated with both entrepreneurship of innovation, as well as persistent gender gaps. 

Entrepreneurship is often associated with \textit{``greater flexibility in terms of an agent's discretion over the length, location and scheduling of their work time''}\footnote{\cite{Quinn1980}}. It is thus expected for people with poor health or disabilities to have a higher probability of self-employment, as a way to avoid workplace discrimination\footnote{\cite{Quinn1980}}. The relation between self-employment and ill-health or disability could be an important control for self-employment decisions, one to be explored with more generous data. Research\footnote{ Works by \cite{BurkeFitzroyNolan2002}; \cite{GeorgellisWall2005} or \cite{CowlingTaylor2001} do not treat the number of children an agent has as the covariate of interest, but use it as a control.} also finds that children act as a greater impediment on a female's entrepreneurial career than that of a male, a factor our model does not control for due to data limitations. In the future, the role of children on parents' decisions to become entrepreneurs, as well as quit self-employment could be studied to capture whether changes of societal perceptions on parenthood have occurred, or whether traditional parenting patterns are still in place.  

 