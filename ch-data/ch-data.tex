\chapter{Data\label{ch:data}}

\section{Data Sources}

Entrepreneurship research has gained considerable momentum in the United States, with some of its world-famous climates for innovation making for innumerable case studies and discussions of replicability. Yet, data collection has been arguably slow to follow, with visible limitations in the comprehensiveness of entrepreneurship measures and their temporal reach. Available data sources capture either business formation in the aggregate, some examples being the Current Employment Statistics (C.E.S)\footnote{Data is collected by the U.S. Bureau of Labor Statistics by monthly sampling from Unemployment Insurance tax accounts.} and the Statistics of U.S. Business (S.U.S.B)\footnote{Data is collected by the U.S. Census Bureau and contains the number of firms, their employment, and annual payroll, categorized by location and industry.}, or agent-level data for which self-employment status is recorded at a given time. There is currently no interplay between the two, limiting the possibilities for a comprehensive approach onto what drives business formation, and what guarantees success in the long term. 

Our interest lies in the socio-economic determinants for business entry decisions, the latter type of data offering the most inclusive perspective on individual incentives. To this regard, the federal government is one of the few sources for self-employment data, operating via the Census Bureau’s American Communities Survey, or the Bureau of Economic Analysis'(BEA)\footnote{\cite{BEA}} Form 1024 filings\footnote{The Form 1024 Schedule C filing is reported annually by individual businesses and proprietors, including information on self-employment status. }\footnote{\cite{Goetz2008}}. U.S. states collect minimal data on self-employed residents (who make at least 10\% of their work population and at most, 18-20\%), another alternative being the Regional Economic Information System under the BEA\footnote{\cite{Goetz2008}}. Although widely used in empirical research\footnote{Works include \cite{Glaeser2007}, \cite{AcsArmington2006}, or Shrestha et al. (2007). }, these sources lack information on entry and exit dynamics, making for a  static measure of self-employment activity. In what follows, we introduce the Kauffman Index of Entrepreneurial Activity, as a leading source on U.S. business formation, and as our self-employment dataset of choice. 

\section{Kauffman Index}
The Kauffman Index is computed from data matched from the Current Population Survey (CPS) conducted by the U.S. Bureau of the Census. An important source compiled by Fairlie (2007)\footnote{\cite{Fairlie2007}}, its microdata contains information on business creation at the owner level, offering a dynamic approach on self-employment decisions. The research unit of the Kauffman Foundation\footnote{The Kauffman Foundation is an American non-profit whose research activity aims to understand what drives innovation and its contribution to economic growth. } identifies all individuals aged twenty to sixty-four who do not own a business in the first survey month, and matches them with the CPS survey for the subsequent month\footnote{\cite{Fairlie2007}}. This generates a two-month pair, capturing whether the same individual started a business with fifteen hours or more worked in the second survey month\footnote{\cite{Fairlie2007}}. Besides self-employment status, demographic variables, geographic and socio-economic indicators are recorded at both survey times of the given year. 

The Kauffman data stands out over other measures of entrepreneurship, given its timely capture of new business formation, recording of all types of business activity and the 15-hours worked condition that implies a high level of owner commitment\footnote{\cite{Fairlie2007}}. These traits makes the Kauffman Index an improved data source on incipient entrepreneurial activity at the U.S. level, capturing the behavioral change associated with an agent’s decision to start a business. By virtue of its origins in the Current Population Survey, it provides a comprehensive list of individual-level variables that contextualize this entrepreneurship decision, serving as push or pull factors in our empirical model. Geographic granularity is given at the region, state and metropolitan area level, with a detailed description of available controls is offered in Table [...]. We use the latest release of the microdata, offering a matched version of the Current Population Survey for the 1998-2014 interval. The dataset contains over 9 million observations, roughly 650,000 for each year in the given period.

The Kauffman Index aligns with the measure for self-employment offered by the Bureau of Economic Analysis, their comparison invoking a statistically significant correlation of $0.513^{**}$\footnote{\cite{Goetz2008}}. In the context of the necessity/opportunity distinction in entrepreneurial activity, the Kauffman Index allows for a differential treatment of entry decisions, with a control for unemployment status in the first survey month. This makes it easy to separate opportunity from necessity entrepreneurship, an important step in isolating the determinants for each type of activity. It is relevant to note that due to its focus on nascent entrepreneurship, and ability to capture individual motivations, the Kauffman Index does not offer a method of tracking down a specific firm’s trajectory, or of separating high-impact entrepreneurship from low-impact businesses. Nationally representative, temporally comprehensive and with a sample size pertaining to the big-data regime, the Kauffman Index data is still a uniquely qualified source measuring entrepreneurial decisions and start-up creation at the U.S. level. 

\section{State Level Controls}
We add economic indicators and political controls to the Kauffman Index microdata, with matching performed by state and year for the 1998-2014 observations. 

\textbf{Economic Output.} Gross Domestic Product (GDP) is one macroeconomic indicator used to measure whether a time period experienced upturn or downturn in economic activity. We matched state-level GDP change from the previous year to the Kauffman Index data, with the measure of choice being per capita real GDP. Because states vary in population count and level of economic output, this option was the most precise measure of economic activity for our empirical needs. The GDP data covers the 1998-2014 period and was collected by the U.S. Bureau of Economic Analysis. Matching was performed according to an agent’s state and year the observation was recorded.  

\textbf{Unemployment.} The average annual unemployment rate was matched for every agent according to state of residence, using data from the U.S. Bureau of Labor Statistics\footnote{The Local Area Unemployment Statistics (LAUS) program is a federal-state joint effort to collect monthly estimates of employment for approximately 7,500 areas, including state-level statistics. The definitions underlying LAUS data originate in the Current Population Survey (CPS),  the main source for the national unemployment rate. 
}. In calculating this rate, subjects are classified as unemployed if they:
\renewcommand{\labelenumi}{\roman{enumi}}
\begin{enumerate}
    \singlespacing
    \item are part of the labor force,
    \item do not have a job, and
    \item have actively looked for one in the previous four weeks (U.S. Bureau of Labor Statistic). 
    
\end{enumerate}


For every observation in the Kauffman data, the unemployment rate for the previous year was also computed using state-matching, with an extra variable generated to capture changes in unemployment from $year_{i-1}$ to $year_{i}$. 

\textbf{Partisanship.} We added the party affiliation of the current governor for every individual observation according to state of residence and given year, using historical data from the National Governor’s Association (NGA)\footnote{The National Governors Association (NGA) is a bipartisan organization for U.S. governors, providing policy research assistance and other services. NGA also holds detailed historical records of U.S. state governance, used for the purpose of our analysis.}. The categories were 

\renewcommand{\labelenumi}{\roman{enumi}}
\begin{enumerate}
\singlespacing
\item Democrat, $dem\_gov$
\item Republican, $rep\_gov$  
\item Independent, $indep\_gov$.
\end{enumerate}

A variable $gov\_change$ designating partisanship change was computed to capture the years where a switch from one label to the other occurred in a given state. 

\textbf{Tax Credits.} We completed the set of control variables with the addition of tax credit data from the Matrix of States with Angel tax Credit Programs developed by the Connecticut Technology Council\footnote{\cite{Nwosu2010}}. This source measures whether a given state benefited from a tax credit program for investors funding new companies in a given year, and were matched to individual observations in the Kauffman data.

A proxy for local policy measures that facilitate entrepreneurship, the presence of tax credits is heterogeneous at the U.S. level, with summary statistic indicating only twenty states are pursuing active steps to encourage private investment in new firms\footnote{\cite{Nwosu2010}}. Some of the ones that do, as in the case of New York, Maine or North Dakota have been pursuing this type of policy for at least a decade, while others like Wisconsin or Kentucky only allow it for a given period or funding amount\footnote{\cite{Nwosu2010}}. 


