The current paper models the determinants of entrepreneurship entry decisions at the U.S. level, asking whether and how men and women respond differently to factors of interest. The probability of someone becoming an entrepreneur is estimated via a Probit specification, as a function of socio-economic factors, demographic controls, local business climate, liquidity constraints and occupational choice. At the individual level, liquidity remains an important factor, along with skill versatility, education and minority status. When it comes to external factors, we find women are incentivized to a greater extent by tax credits and the local political climate, while men respond to positive changes in the local economy. and Across industries, female entrepreneurs cannot use self- 

\textbf{Keywords:} gender, entrepreneurship, innovation, economic growth, incentives, asymmetry, labor-market discrimination 
