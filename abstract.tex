\singlespacing 

The current paper models the determinants of entrepreneurship entry decisions at the U.S. level, asking whether and how men and women respond differently to factors of interest. The probability of someone becoming an entrepreneur is estimated via a Probit Model, as a function of socio-economic factors, demographic controls, local business climate, liquidity constraints and occupational choice. For every model, errors are clustered by metropolitan area, and coefficient equality is tested across gender samples. At the individual level, we find that availability of capital remains an important factor, along with skill versatility, education and minority status. When it comes to external factors, we find women are incentivized to a greater extent by tax credits and the state of the political climate, while men respond to positive changes in the local economy. Across industries, we find patterns of women being underrepresented in STEM fields, which is consistent with wage employment and disproves our hypothesis of a breaking-barriers mechanism to starting one's business. 

\singlespacing \textbf{Keywords:} gender, entrepreneurship, self-employment, innovation, economic growth, incentives, asymmetry, female labor force participation.

\bodyspacing