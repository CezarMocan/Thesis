\chapter{Policy Implications\label{ch:policy}}

\section{The Case for Innovation Policy}

Nascent enterprises on the path to becoming important job creators are of great interest to policymakers. But why is it important for policy to intervene in the equilibrium of births and deaths of companies, and why the focus on the self-employed? Authors that study economic clustering and its effects argue that innovation-oriented policy helps internalize externalities, for at least two reasons: 

\renewcommand{\labelenumi}{\roman{enumi}}
\begin{enumerate}
\item It pays off in the long term to help new firms and incentivize them to remain active, as they will pay taxes for years to come. (Chatterji et al. 2014) 
\item Spillovers can increase productivity in the aggregate, leading to human capital externalities. (Moretti, 2004). 
\end{enumerate}

It is thus justified to foster a culture of innovation and business formation, with hopes for regional development in the long run. There are also redistributive goals to encouraging business formation in specific areas, with measures that serve as example of how to "help poor places instead of poor people" (Chatterji et al. 2014). 

Last but not least, innovation policy can be a means to combat market imperfections, as well as discriminatory factors that might prevent access to funding of specific areas or groups of people (Chatterji et al. 2014). With our findings indicating a persistent gender gap in self-employment, as well as minorities being underrepresented in business ownership, there is legitimacy to policies meant to encourage nascent entrepreneurship in marginalized groups. The idea of an economic and societal cost resulted from marginalizing women in the global economy is not new (Bakker, I. (1994); Blank, (1993); Boserup, (1970)), but the  gender integration could become a powerful tool for developing entrepreneurial solutions.  

\section{Findings in Context}

Barriers at the entry stage of the entrepreneurship experience still exist, despite the United States making for one of the most favorable environments in this respect (Autio, 2007). As we saw reflected in our coefficient estimates, women and most minority groups still face significant obstacles to entry. At the same time, men and women respond differently to monetary, lifestyle, policy or economic factors, in ways that might not be incorporated by policy initiatives. In line with this finding, economic literature points out that entry barriers also transfer to the expansion phase for minority-owned businesses, which is costly to productivity at the U.S. level (Reynolds and White, 1997). 

\subsection{Funding}

As Table A.1 in [Descriptive Statistics] indicates, the Small Business Administration\footnote{The Small Business Administration was created to provide incentives for small businesses at the regional level, in the form of loans, grants and local assistance (U.S. Small Business Administration website).} has in recent years provided billions of dollars in loans, with values stabilizing around 7 billion a year. We note the persistent gender gap in funding, which could be explained to a great extent by asymmetries in self-employment entry across the two groups, as indicated by our results. Along with that, most funding goes to existing businesses (72\% on average) as opposed to new ones (28\%), and the majority of SBA funds are directed to white-owned firms (over 60\%), which further reflects the distribution of the self-employed. 

With our coefficient estimates indicating a strong positive effect for availability of capital on the likelihood of self-employment, more attention to nascent businesses and their capital constraints from federal organizations such as the SBA could help promote business formation. 

\subsection{Barrier Reduction}

Apart from initiatives meant to provide assistance and funding to nascent small businesses, U.S. cities have in the past implemented policies meant to reduce entry barriers for particular groups, an example being local contracts set aside for businesses owned by women and racial minorities (Chatterji et al., 2014). This in turn, is said to have narrowed the black-white self-employment gap in the 80s by more than 3\% (Chatterji et al. 2014).

Given the increasing share of the total population minorities represent in the U.S., policy that targets business formation among these groups will need increasing attention. One can imagine local programs like the 1980s initiatives granting local contracts to women and minority owned businesses getting more traction given that the self-employment gap between men and women, or blacks and whites, does not show signs of decreasing.  

\subsection{Local Supply of Entrepreneurs}

Some policies are meant to increase the so-called "local supply of entrepreneurs", addressing issues of location, education or the availability of information (Chatterji et al., 2014). The tendency for the self-employed population "to disproportionately found firms near where one was born" thus challenges policy to ensure the availability of educational programs or science and technology initiatives at the local level, along with ensuring access to basic business knowledge (Chatterji et al. 2014).  Adding to this, 42 states currently require entrepreneurial education, more than double from 19 in 2009 (Chatterji et al. 2014). 

Given that skill versatility was found to be one of the most important factors in modeling the transition to self-employment, the ability to change industries and market oneself as a "jack of all trades" could be of valuable importance, and more attention must be paid to the degree to which certain regions allows or impede these labor market transitions.

In increasing the supply of local entrepreneurs, high-skilled immigration is also seen as a key issue to be addressed at the federal level. STEM policy is often paired with measures to foster entrepreneurship, and proposals for an entrepreneurial visa providing a path to permanent residency for foreign investors depending on the number of jobs created have been long speculated (Chatterji et al. 2014). However, our findings indicate that 


Single people - how about incentivizing them? \\
The nuance to high skilled immigrants - yes, 1\% found uber, but how do you incentivize the rest? more certainty? 















