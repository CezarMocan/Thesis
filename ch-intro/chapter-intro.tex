\chapter{Introduction\label{ch:intro}}

\begin{flushright}
    \textit{"For many commentators this is the era of the entrepreneur." } \\
    (Robert Goffee and Richard Scase)
\end{flushright}


Entrepreneurship: it has become an almost commonplace preoccupation in American society, treated with great interest by both the general public, and specialized literature---with more and more consensus regarding its beneficial impact for employment, innovation and growth for the economy at large\footnote{\cite{ReynoldsWhite1997}}. With more than 10\% of the U.S. population working for their own company at a given time\footnote{Authors \cite{bucks2006recent} invoke this makes for 13 million business owners holding an astounding 37.4\% percent of total U.S. wealth. }, and another large proportion indicating they would want to pursue this path\footnote{\cite{kennickell2006currents}}, it is without doubt a topic worthy of increased attention from policy makers. Nonetheless, the calls for policy action to expand on U.S. entrepreneurial activity via investment, research and development, or educational programs are frequently substantiated by an incomplete understanding of the process of  firm creation.  The focus is often on the new entity and its subsequent success, with little consideration for the individual agent as the major driver of economic value in a new organization\footnote{\cite{BlanchflowerOswald1998}}. But isn't that one of the main appeals, and promises of entrepreneurship---that of creativity and independence, of empowering individuals?

When the individual level is considered, one of the most significant differences pertaining to
entrepreneurship is noted in the proportion of men and women who choose to pursue business formation. Studies bluntly conclude that women are less likely to start their own business\footnote{\cite{kennickell2006currents}}\hspace{.15em}\footnote{\cite{PatrickStephensWeinstein2016}}\hspace{.15em}\footnote{\cite{koellinger2013gender}}, or less likely to apply for business loans\footnote{\cite{SBA}}, this in turn leading to profound implications for aggregate productivity, regional growth and gender parity in the labor market\footnote{\cite{ReynoldsWhite1997}}. Yet women's share of both self-employment and overall employment is on a continuous rise\footnote{\cite{PatrickStephensWeinstein2016}}, making the study of the factors affecting these decisions particularly important to our understanding of the U.S. labor force and its economic prospects. As increases in entrepreneurship and self-employment have been proven to be closely followed by economic growth\footnote{\cite{ReynoldsWhite1997}}, it is especially important to capture what incentivizes women to become self-employed and let such insights make for policy that caters to the economic agent, and not just the enterprise.

In what follows, we raise the question: what factors drive entrepreneurship entry across genders, and how do motivations for business formation differ for men and women? Specifically, we are interested in whether men and women respond differently to macroeconomic conditions and policy environment variables, and whether expectations about skill level, education and lifestyle changes hold across the two groups\footnote{The author of the current paper does not subscribe to the idea of a gender binary. However, given the limitations of current data sources in capturing more than two gender identities, most research focuses on the two categories invoked, which we shall refer to for the purposes of this analysis as well}. Previous literature studying the entrepreneurial process includes Fuchs (1982)\footnote{\cite{fuchs1980self}}, Evans and Jovanovic (1989)\footnote{\cite{EvansJovanovic1989}} and more recently, Blanchflower and Oswald (1998)\footnote{\cite{BlanchflowerOswald1998}}, Cowling and Taylor (2001)\footnote{\cite{CowlingTaylor2001}} and Patrick et al. 2016\footnote{\cite{PatrickStephensWeinstein2016}}.  While the current paper abides by the general spirit of these investigations by looking at socio-economic and environmental variables affecting self-employment, it diverges in both the data selection and methodology used. 

One key distinction from most previous empirical work is that, by looking at the \textit{supply side} of entrepreneurship, we bring the focus on the determinants of self-employment \textit{entry} at the U.S. level, and not just self-employment status. We further ask whether---and how---men and women respond differently to factors of interest, modeling determinants of \textit{decisions} and not merely of firm \textit{states}. For that, we seek to estimate the probability  of an agent becoming an entrepreneur via a function of socio-economic factors, demographic controls, local business climate, liquidity constraints and occupational choice---for which we implement a Probit Model. 

In doing so, we recognize the importance of the local, social context: the family, the network of friends and industry acquaintances, the local customers, the regional institutions and policies, local economy or entrepreneurial climate, even regional ideologies and local cultural differences. We thus cluster standard errors by metropolitan area in an attempt to respect this aspect of asymmetric locality within our analysis. Yet above all, we acknowledge the relevance of \textit{gender} within this social context. And this serves as our main hypothesis for this paper: that gender, in all its complexities,  has a significant effect on the likelihood of business formation. This thus motivates our additional implementation of a data model with differentiated samples for men and women to be contrasted along the general results. 

In what follows, we discuss the data choices for modeling an agent’s probability of starting a business, followed by a thorough description of our methodology and expectations for coefficient estimates upon implementing the Probit Model.  We then report our results for the entire dataset, as well as for the male and female samples, with the specification of whether effects differ significantly across genders. We discuss these results in depth and related to our initial hypotheses, delineating the possible mechanisms behind them. We substantiate our analysis with a discussion of entrepreneurship policy, relating our findings to past and current initiatives, all the while framing possible directions for future empirical work. 
